\documentclass[titlepage]{article}
\title{Final CW Assignment}
\author{Sadra Sefidgari}

\begin{document}
\maketitle

\section{Git and GitHub}
\subsection{Repository Initialization and Commits}
I started with creating a repository on github and cloning it into local machine.
Then, downloaded the GitHub Actions related files from the assignment repo and copied it into my cloned repo.

After that , I created a test main.tex file to see if it works .
I first compiled it on my local machine and then used .gitignore to ignore those files created by MikiTex.

I also added a tag and pushed the commit into GitHub and the workflow seems to be working just fine.
\subsection{GitHub Actions for LaTeX Compilation}

As I wrote in the last subsection,I just copied and pasted it into my repo , so not much of challenge.
Before I did that , I edited the settings of actions to read and write, making sure that the action would run perfectly.

I also needed to know how to add tags. I searched it up and it did not contain anything concerning that would result in a bigger challenge later on.

\section{Exploration Task}
\subsection{Vim Advanced Features}
Let's checkout some cool features.

Custom Command: Vim Allows us to create user-defined command for certain things. For example there is a very long sequence of commadns you need to do every time.
Vim allows you to combine all of them in a user-defined command for ease.In Nvim, you can also script in lua for a wider amount of possiblities with the command-defining matter.

Marks: Think of them as bookmarks. You specify a location in a file , probably a lot longer than anyone could every possibly imagin.Then, you mark a location to don't lose it.

Sort: Vim has a uniqe command to sort test directly inside it!! You just have to tell where to look for the sort by a parrern or just tell it to do it in the alphabetical order!




  
\end{document}
